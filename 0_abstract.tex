%========================================================
% Abstract (English example; <= 200 words target)
%========================================================
Human-following is an important function for service robots, as it supports tasks such as mobility assistance and object
transport. However, many existing systems do not handle cases in which the target person is hidden by obstacles, known as occlusion,
and they often fail to continue tracking once the target is lost. To address this problem, we propose a human-following system that
combines 3D LiDAR with a person re-identification method to enable re-detection and recovery after occlusion. For person detection,
we use PointPillars, which provides fast and accurate inference, and we train it with the TAO Toolkit using a custom dataset of 19,500
point-cloud frames. For re-identification after tracking loss, we apply the ReID3D framework, which can extract strong features from
point clouds and allows robust identification of the target. To evaluate the training effectiveness of the proposed method, a comparative
evaluation is conducted on a custom evaluation dataset consisting of 4,500 frames, using a standard pretrained model as a baseline.
