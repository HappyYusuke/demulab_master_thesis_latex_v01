\section{Conclusion}

In this study, we developed a person-following system using 3D-LiDAR and verified its pedestrian detection performance as well as its tracking and re-identification capabilities. For this system, we adopted PointPillars, optimized for the point cloud characteristics of Livox 3D-LiDAR. By training the model with a custom-built dataset, we constructed a high-precision pedestrian detector. Furthermore, by integrating ReID3D, we achieved a person-following system capable of re-identifying the target person even after occlusion occurs.

To verify the effectiveness of the proposed method, we conducted evaluation experiments using both the custom dataset and a public dataset. As a result, the optimized PointPillars showed pedestrian detection performance that significantly exceeded that of conventional models. Additionally, comparative experiments using TPT-Bench showed that our method demonstrated higher robustness than general RGB trackers, such as SiamRPN++ and MixFormer, in environments with lighting changes and long-term occlusion. On the other hand, compared to state-of-the-art methods like OCL-RPF and DiMP, our results indicated that challenges remain in tracking accuracy and re-identification ability. It became clear that this is caused by the lack of an online model adaptation mechanism in our method and the limits of prediction accuracy due to the linear Kalman filter.

\section{Future Work}

We identify two main areas for future work.

First is the further improvement of pedestrian detection performance. The results of our experiments confirmed a tendency for missed detections when pedestrians stop and stand upright. The likely cause is that the dataset used for training was dominated by point clouds of moving pedestrians, and there was a lack of data for stationary states. Therefore, we expect to improve detection performance during stops by constructing a new dataset with more point clouds of stationary pedestrians and re-training the model.

Second is the advancement of tracking and re-identification algorithms. Regarding tracking, we confirmed the limitations of the currently adopted linear Kalman filter. Prediction errors tend to occur in situations where the relative trajectory becomes non-linear due to the difference in movement between the pedestrian and the robot. Therefore, it is necessary to introduce non-linear estimation methods, such as the Extended Kalman Filter (EKF) or Unscented Kalman Filter (UKF), or to consider methods that directly regress the bounding box from observation data, like DiMP. Regarding re-identification (ReID), improving the adaptability of the target model is essential. The discussion of the experimental results suggested that the performance gap with OCL-RPF is largely due to the presence or absence of online continuous learning. Since the current pre-trained fixed model cannot sufficiently handle posture changes and changes in viewing angle during tracking, we need to introduce a mechanism (online learning mechanism) that dynamically updates the target's feature model using time-series data.