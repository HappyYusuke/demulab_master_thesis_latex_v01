\section{まとめ}
本研究では、3D-LiDARを用いた人追従システムの開発を行い、歩行者検出性能および追跡・
再識別性能の検証を行った。本システムでは、Livox社製3D-LiDARの点群特性に最適化を施
したPointPillarsを採用し、独自に構築したデータセットで学習を行うことで高精度な歩行
者検出器を構築した。さらに、ReID3Dを統合することで、オクルージョン(遮蔽)発生後も
対象人物を再識別可能な人追従システムを実現した。

提案手法の有効性を検証するため、独自データセットおよび公開データセットを用いた評価
実験を行った結果、最適化を施したPointPillarsが従来モデルを大幅に上回る歩行者検出性
能を示した。また、TPT-Benchを用いた比較実験の結果、本手法は照明変化や長期的なオク
ルージョンが発生する環境下において、SiamRPN++やMixFormerといった汎用RGBトラッカー
を上回る頑健性を示した。一方で、最先端の手法であるOCL-RPFやDiMPと比較すると、追跡
精度および再識別能力において課題が残る結果となった。これは、本手法におけるオンライ
ンでのモデル適応機構の欠如や、線形カルマンフィルタによる予測精度の限界に起因するこ
とが明らかとなった。

\section{今後の課題}
今後の課題として、以下の二点が挙げられる。

第一に、歩行者検出性能のさらなる向上である。本実験の結果より、歩行者が停止し直立
している場合に検出漏れが発生する傾向が確認された。この原因として、学習に使用した
データセットに歩行者が動作している状態の点群が支配的であり、静止状態のデータが不足
していたことが考えられる。したがって、歩行者が停止している状態の点群を拡充したデー
タセットを新たに構築し、再学習を行うことで、静止時における検出性能の改善が見込まれ
る。

第二に、追跡および再識別アルゴリズムの高度化である。 まず、トラッキングに関しては、
現在採用している線形カルマンフィルタの限界が確認された。歩行者やロボットの挙動差に
より相対軌跡が非線形となる場面では予測誤差が生じやすいため、拡張カルマンフィルタ
 (EKF) やアンセンテッドカルマンフィルタ (UKF) といった非線形推定手法の導入、あるい
 はDiMPのように観測データから直接矩形を回帰する手法の検討が必要である。 また、再識
 別(ReID)に関しては、ターゲットモデルの適応能力向上が不可欠である。実験結果の考察
 において、OCL-RPFとの性能差はオンライン継続学習の有無に大きく起因することが示唆さ
 れた。現在の事前学習済みの固定モデルでは、追跡中の姿勢変化や観測アングルの推移に
 十分に対応できないため、時系列データを用いてターゲットの特徴モデルを動的に更新す
 る仕組み(オンライン学習機構)の導入が求められる。