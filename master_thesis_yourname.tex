% KIT 出村研究室 修士論文とAdvanced Robotics投稿用 latexファイル 2026-01-07

%%%%% 使い方 %%%%%
% 修論の場合は何も変更しないでコンパイルする。
% AR投稿論文は次行のthesisをarに変えてコンパイルする。
\def\KITMODE{ar} % ← ar にしたければ thesis を ar に変える
%\def\KITMODE{ar} % ← 修論にしたければ ar を thesis に変える
\PassOptionsToPackage{dvipdfmx}{graphicx}
\documentclass[\KITMODE]{kitinteract} 
%%%%%%%%%%%%

\usepackage[caption=false]{subfig}% Support for small, `sub' figures and tables
\usepackage{amsthm}
\usepackage{amsmath}
\usepackage{booktabs}
\usepackage[numbers,sort&compress]{natbib}% Citation support using natbib.sty
\usepackage{geometry}
\geometry{margin=25mm}
\usepackage{enumitem}
\usepackage{hyperref}
\usepackage{setspace}
\usepackage{comment}
\usepackage{amssymb}   % for \square

% By kanazawa
\usepackage{booktabs}
\usepackage[hang,small,bf]{caption}
%\usepackage[subrefformat=parens]{subcaption}
\captionsetup{compatibility=false}
\usepackage{indentfirst}
\usepackage{pgffor}
\usepackage{float}

%\usepackage[nolists,tablesfirst]{endfloat}% To `separate' figures and tables from text if required
%\usepackage[doublespacing]{setspace}% To produce a `double spaced' document if required
%\setlength\parindent{24pt}% To increase paragraph indentation when line spacing is doubled

\bibpunct[, ]{[}{]}{,}{n}{,}{,}% Citation support using natbib.sty
\renewcommand\bibfont{\fontsize{10}{12}\selectfont}% Bibliography support using natbib.sty
\makeatletter% @ becomes a letter
\def\NAT@def@citea{\def\@citea{\NAT@separator}}% Suppress spaces between citations using natbib.sty
\makeatother% @ becomes a symbol again

\theoremstyle{plain}% Theorem-like structures provided by amsthm.sty
\newtheorem{theorem}{Theorem}[section]
\newtheorem{lemma}[theorem]{Lemma}
\newtheorem{corollary}[theorem]{Corollary}
\newtheorem{proposition}[theorem]{Proposition}

\theoremstyle{definition}
\newtheorem{definition}[theorem]{Definition}
\newtheorem{example}[theorem]{Example}

\theoremstyle{remark}
\newtheorem{remark}{Remark}
\newtheorem{notation}{Notation}


% ----- Metadata shared by both modes -----
\kituniversity{Kanazawa Institute of Technology}
\kitdegree{Master's Thesis}
% Default values are already set in kitinteract.cls, but you can override them here if needed.
\kitfaculty{Graduate School of Engineering}
\kitdepartment{Mechanical Engineering}
%\kitmajor{Graduate Program in Robotics}

% ----- KIT emblem on the cover (thesis mode only) -----
% Place KIT_kosho.eps in the same folder as your .tex file.
% You can adjust the size with \kitlogowidth{<dimension>}.
\kitlogo{KIT_kosho.eps}
% \kitlogowidth{28mm}

% ----- 修論用 -----
\kittitle{Your Master's Thesis Title}
\kitauthor{Your Name}
%\kitstudentid{20XX0000}
\kitsupervisor{Prof. Kosei Demura}
\kitsubmissiondate{January 2026}

% ----- AR用 -----
\title{Development of a Human Following System Using 3D LiDAR and ReID3D}
\author{ % 指示がなければ出村はラストオーサー
\name{Yusuke Kanazawa\textsuperscript{a} and Kosei Demura\textsuperscript{b}\thanks{CONTACT Kosei Demura. Email: demura@neptune.kanazawa-it.ac.jp}}
\affil{\textsuperscript{a}Graduate School of Mechanical Engineering, Kanazawa Institute of Technology, Nonoichi, Ishikawa, Japan; \textsuperscript{b}Department of Robotics, Kanazawa Institute of Technology, Nonoichi, Ishikawa, Japan}
}

% ----- (Optional) One-page front note (inserted BEFORE cover/title) -----
% This page uses a tighter line spacing and the same formatting in both modes.
%\frontnote{frontnote.tex}


% ----- Abstract & Key words (write once; used in both modes) -----
\kitabstract{%
%========================================================
% Abstract (English example; <= 200 words target)
%========================================================
Human-following is an important function for service robots, as it supports tasks such as mobility assistance and object
transport. However, many existing systems do not handle cases in which the target person is hidden by obstacles, known as occlusion,
and they often fail to continue tracking once the target is lost. To address this problem, we propose a human-following system that
combines 3D LiDAR with a person re-identification method to enable re-detection and recovery after occlusion. For person detection,
we use PointPillars, which provides fast and accurate inference, and we train it with the TAO Toolkit using a custom dataset of 19,500
point-cloud frames. For re-identification after tracking loss, we apply the ReID3D framework, which can extract strong features from
point clouds and allows robust identification of the target. To evaluate the training effectiveness of the proposed method, a comparative
evaluation is conducted on a custom evaluation dataset consisting of 4,500 frames, using a standard pretrained model as a baseline.

}
\kitkeywords{Human Following, Person Tracking, Person Identification, Point Clouds, 3D LiDAR}


% ----- (Optional) Advanced Robotics metadata -----
% In ar mode, \kittitle sets \title and \kitauthor sets \author.
% You can add journal-style items like affiliations/thanks per interact.cls.

\begin{document}

% Unified front matter:
%  - thesis mode: cover + abstract + ToC/LoF/LoT
%  - ar mode: title + abstract + keywords
\makefrontmatter

\mainmatter

\chapter{Introduction}
\noindent In recent years, there has been a strong expectation for the social implementation of autonomous mobile robots in diverse fields such as logistics, healthcare, and security. The aim is to resolve labor shortages and improve operational efficiency. For these robots to coexist with humans and perform tasks cooperatively, the "person following function"—which recognizes and follows a specific individual—is an extremely important element technology. However, in real-world environments where the surroundings change dynamically, situations frequently occur where the target person temporarily disappears from the sensor's field of view. This is often caused by other pedestrians crossing the path or obstacles blocking the view (a phenomenon known as occlusion). To ensure the continuity of tracking, "Re-identification" is indispensable. This process correctly determines whether a person who has reappeared is the original target when the robot recovers from an occlusion.

To realize person following systems in real-world environments, various approaches have been attempted so far. In methods using cameras (RGB images), systems combining deep learning-based person detection and tracking algorithms have been widely studied. Since these methods provide rich texture information, they demonstrate high identification performance under good lighting conditions. However, they strongly depend on appearance information, such as clothing, and environmental brightness. Therefore, there is a problem that recognition accuracy significantly decreases due to changes in lighting conditions, such as backlight or low light. On the other hand, methods using 2D-LiDAR allow for robust measurement that does not depend on the brightness of the environment, but the information obtained is limited to distance data on a horizontal plane. Generally, tracking is performed by clustering the cross-sections of legs or the torso. However, it is difficult to extract physical features of an individual from only the horizontal cross-sectional shape, making it hard to distinguish one person from another. Regardless of the sensor type, occlusion is unavoidable in environments with crowds or many obstacles. Tracking algorithms based only on motion models, such as Kalman filters, find it difficult to recover after losing the target due to long-term blocking or irregular movement. Therefore, establishing a robust re-identification method that can clearly distinguish the target from surrounding people upon reappearance and resume tracking has become an urgent issue in person following technology.

Therefore, the purpose of this study is to construct a system with high robustness against occlusion by introducing the re-identification model ReID3D \cite{ReID3D} into a person following system using 3D LiDAR. The novelty of this research lies in focusing on "re-identification" based on 3-dimensional point cloud information, rather than just the "detection and tracking" of the target. Many conventional studies on person following using 3D LiDAR have focused on high-precision detection or motion prediction, and there are few cases that focus on personal re-identification using point clouds. ReID3D, adopted in this study, is a person re-identification framework proposed by Guo et al. It can extract geometric features such as height, body shape, and gait from a person's 3D point cloud as high-dimensional vectors. By applying this to a person following system, we aim to realize a system that is not influenced by lighting conditions and can accurately rediscover the target even after an occlusion.

%\section{Thesis Organization}

This thesis consists of five chapters, and the outline is as follows.

In Chapter 2, we comprehensively review conventional research on person detection and person following technologies by mobile robots. Specifically, we systematically organize methods using cameras, 2D LiDAR, and 3D LiDAR, and clarify the advantages based on the characteristics of each sensor, as well as technical issues such as robustness to environmental changes. Furthermore, we explain the basic theories of the deep learning models that form the basis of this research and the system framework adopted. We also clarify the position and uniqueness of this research relative to conventional studies.

In the following Chapter 3, we describe in detail the proposed method constructed in this project to solve the issues raised in the previous chapter. We present specific algorithms for sensor fusion and data processing, as well as the architecture of the entire system. We also present the design philosophy and implementation details of a new person following approach that integrates 3D LiDAR and re-identification technology.

In Chapter 4, we discuss the experiments conducted to demonstrate the effectiveness of the proposed method. After explaining the experimental environment, the set evaluation scenarios, and the comparison methods, we quantitatively analyze the obtained experimental results. Through this, we verify that the proposed system possesses sufficient following performance and robustness in real-world environments.

Finally, in Chapter 5, we summarize the knowledge and achievements obtained in this research. We conclude on the usefulness of the proposed method revealed through a series of verifications, while also organizing the current limitations and remaining issues, and describing the prospects for future research.
\clearpage

\chapter{Related Work}
\section{Person Detection Methods Using 2D LiDAR}

In research using 2D LiDAR, a common approach involves clustering the obtained point cloud data of horizontal cross-sections, detecting them as cross-sectional shapes of legs or torsos, and then performing time-series tracking using methods such as Kalman filters.

Arras et al. \cite{2D LiDAR} proposed a method to robustly detect people even in complex environments using planar scan data obtained from a 2D Laser Range Finder (LRF). Many conventional approaches used pre-defined features to detect and track human legs, but these had the problem of relying on manual design and tuning. In contrast, Arras et al. constructed a high-precision detector adapted to the environment by using an approach that selects and integrates effective features from a large number of candidates through supervised learning.

In their proposed method, the point cloud data obtained from the LRF is first segmented using a "jump distance condition," which divides the data at points where the distance between adjacent beams exceeds a certain threshold. Each obtained segment is treated as a leg candidate, and 14 types of scalar features representing its shape and statistical properties are calculated. Specifically, in addition to static geometric features—such as the number of points composing the segment, standard deviation, mean deviation from the median (which is robust against outliers), jump distance to adjacent segments, segment width, linearity and circularity (fitness to lines and circles using the least squares method), radius of the fitted circle, boundary length, boundary regularity, mean curvature, and mean angular difference measuring convexity—dynamic features such as the mean speed calculated from the difference between consecutive scans are defined.

AdaBoost is adopted for the learning algorithm. By combining these simple features (weak classifiers), a strong classifier is constructed to distinguish between segments corresponding to human legs and those that are not. By using AdaBoost, it becomes possible to automatically learn the most informative features and optimal thresholds from the data so as to maximize classification performance.

Experiments were conducted in corridors and cluttered office environments where both stationary and moving people were present. The performance evaluation of the learned classifier showed that it could detect people with a high accuracy rate exceeding 90\%, even in complex environments with many pieces of furniture and objects. In particular, compared to heuristic methods based on geometric rules often used in conventional research (such as setting fixed thresholds for leg width or jump distance), the proposed method significantly reduced the false detection rate, demonstrating its superiority. Furthermore, it was confirmed that high accuracy was maintained even when applying a classifier trained in one environment to an unknown environment not included in the training data, indicating that the learned model possesses high versatility.

Moreover, through the analysis of the learned classifier, important insights regarding effective features for person detection were obtained. As a result of analyzing the weights of the features selected by AdaBoost, the features contributing most to the classification were not the simple leg width or circularity, but the "radius" of the fitted circle and the "convexity (mean angular difference)" of the segment. This suggests that while the shape of legs in 2D scan data takes various forms and is not necessarily a clean arc or straight line, the sense of size and the degree of outward swelling are the most robust indicators. In addition, experimental results from cases where the dynamic feature "movement speed" was added showed that the improvement in accuracy was slight compared to cases using only static geometric features. From this, Arras et al. concluded that motion information is not essential for robust detection including stationary people, and that sufficiently high-precision detection is possible with just a combination of appropriate geometric features. However, they did not mention re-detection or recovery of tracking after occlusion.



\section{Person Following Methods Using 3D LiDAR}

In research using 3D LiDAR, a common pipeline involves removing ground point clouds from 3D spatial data, extracting point cloud clusters that are candidates for people using Euclidean clustering or similar methods, and then tracking them using Kalman filters or particle filters. Yan et al. \cite{3D LiDAR} proposed a framework where a mobile robot equipped with a 3D LiDAR learns a person detector online while adapting to the driving environment. In person detection using 3D LiDAR, changes in point cloud density due to distance and the human cost involved in creating training data (annotation effort) have been major issues. In response to this, Yan et al. developed a method to automatically retrain a classifier (SVM), which is initialized with a small amount of labeled data, using data obtained during the robot's operation.

The core of the proposed method is an adaptive clustering technique and a training data generation mechanism called "P-N Experts." First, in the segmentation (clustering) of point clouds, to handle the point cloud density becoming sparser with distance from the LiDAR, adaptive clustering is adopted. This sets different distance thresholds for concentric zones centered on the sensor. This achieves higher precision segmentation than conventional methods based on Euclidean distance or depth image-based methods, in both dense point clouds at close range and sparse point clouds at long range.

For person classification, an SVM classifier using geometric features is employed. However, to specifically improve detection performance at long distances, "Slice features" are newly introduced. This involves dividing the point cloud into 10 vertical slices and describing the width and depth of each slice. This enables robust feature extraction even for distant people with few data points.

In the online learning process, the results of multi-target tracking (UKF) are used to automatically correct the errors of the classifier. Specifically, the "P-expert" uses the continuity of tracking trajectories to pick up missed detections (False Negatives) and adds them as positive examples, while the "N-expert" identifies false detections (False Positives) such as stationary objects and adds them as negative examples. By periodically retraining the classifier using the training data automatically generated in this way, it is possible to adapt to the characteristics of people and backgrounds specific to the environment without additional manual annotation.

Results of evaluation experiments using a dataset acquired in a large-scale indoor environment (L-CAS 3D Point Cloud People Dataset) confirmed that the proposed adaptive clustering showed higher segmentation accuracy compared to conventional methods. It was also confirmed that the classifier updated by online learning achieved a higher F-score (the harmonic mean of precision and recall) than a classifier trained manually offline. Yan et al.'s research is an important achievement showing the effectiveness of geometric features in person detection using 3D LiDAR and the utility of self-supervised learning by incorporating tracking information into the feedback loop. However, they did not mention re-detection or identification after occlusion.



\section{Occlusion-Aware Person Following Methods Using Monocular Cameras}

Person following systems using cameras (RGB images) are characterized by their ability to utilize rich optical information (Photometric features) such as the target's color and texture. In early studies, detection methods using color histograms or HOG (Histogram of Oriented Gradients) features were mainstream, but in recent years, methods combining Deep Learning and tracking algorithms have been widely researched. Additionally, approaches that incorporate online learning to adapt to the environment and target, rather than using pre-trained models as is, have also been proposed. Koide et al. \cite{Monocular} proposed a mobile robot person following system using only a "monocular camera," which is cheaper and easier to introduce compared to LiDAR or RGB-D cameras.

In person following using a monocular camera, there are mainly two technical challenges. The first challenge is "how to accurately estimate the target's 3D position from an image where depth information cannot be obtained." The second challenge is "how to realize robust tracking in real-time on a robot with limited computational resources, despite lighting variations and pose changes." Conventionally, methods using HOG features or color histograms with low computational cost (such as KCF) were fast but vulnerable to lighting changes and blending into the background. On the other hand, methods using deep learning (CNN) have high discrimination capability but extremely high computational load, making real-time operation on mobile robots equipped with embedded GPUs difficult.

To address these issues, Koide et al. attempted to solve them by integrating the following two approaches.

First, for the challenge of 3D position estimation, they introduced a tracking framework combining deep learning-based skeleton detection (OpenPose) and geometric constraints. Specifically, focusing on the positions of the "ankles" and "neck" of the person detected in the image, they combined the assumption that these exist on the Ground Plane with height estimation by a Kalman filter (UKF). This achieved high-precision position estimation in the robot coordinate system without using distance sensors. This method has the advantage of being able to continue tracking using the neck position information even when the feet are hidden by obstacles (occlusion).

Second, for the trade-off between computational cost and robustness, they proposed a method called "On-line Deep Feature Selection (ODFS)." Instead of using all the vast feature maps (Convolutional Channel Features: CCF) obtained from a VGG-16 model pre-trained on ImageNet, this method dynamically selects only the features most effective for separating the "target" from the "background" in the current frame using online boosting. This allows for adaptation, such as prioritizing color features when lighting conditions change or edge features when the shape is distinctive. This achieved real-time processing of 20fps on average on an embedded computer (NVIDIA Jetson TX2) while utilizing the expressive power of deep learning.

In evaluation experiments, verification was conducted in indoor and outdoor scenarios involving lighting changes, complex backgrounds, target rotation, and occlusion. As a result, the proposed method recorded the highest tracking success rate and accuracy compared to existing trackers like KCF and TLD. It demonstrated that robust person following in real environments is possible through the efficient use of deep learning, even with the limited sensor configuration of a monocular camera. On the other hand, this method using a monocular camera leaves the issue that direct measurement of depth information is fundamentally difficult and the field of view is limited. In the experiments, although it showed high following performance under specific conditions, conditions leading to mis-estimation or detection failure were reported, such as detection failure when the target was too close to the camera or reduced distance estimation accuracy at long distances, which are due to sensor characteristics.



\section{PointPillars}
Lang et al. \cite{PointPillars} proposed a new network architecture "PointPillars" aimed at balancing inference speed and detection accuracy in 3D object detection tasks, which are extremely important in autonomous driving and robotics.

Conventionally, for 3D object detection using point cloud data, there were methods that voxelized the point cloud and applied 3D Convolutional Neural Networks (3D CNN), or methods that projected the point cloud onto a Bird's Eye View (BEV) and applied 2D CNN. The former boasted high accuracy but had very high computational costs (e.g., VoxelNet), while the latter was fast but had the problem of losing 3D shape information, resulting in inferior accuracy. Also, methods that process point clouds directly, like PointNet, had good computational efficiency but issues with scalability for large-scale point clouds. In contrast, PointPillars adopts a new encoder that organizes point clouds into vertical columns (Pillars) for processing. This realizes a fast and high-precision detection pipeline composed only of standard 2D convolutions, without using any 3D convolutions.

The architecture consists of three main parts: the Pillar Feature Net (PFN), the Backbone (2D CNN), and the Detection Head (SSD). First, the Pillar Feature Net (PFN) plays the role of converting raw point cloud data into a "Pseudo-image." The input point cloud is first divided into vertical "Pillars" based on a grid on the $xy$ plane. Points within each pillar are represented as 9-dimensional feature vectors containing coordinates $(x, y, z)$, reflection intensity $r$, offset from the geometric center $x_c, y_c, z_c$, and offset from the pillar center $x_p, y_p$. By applying a simplified PointNet (Linear layer, Batch Norm, ReLU) to this, the features of each point are mapped to a high-dimensional space. Furthermore, by performing Max Pooling for each pillar, a feature vector representing each pillar is extracted. Through this process, the point cloud data is converted into a 2D pseudo-image with the format $(C, H, W)$. This method differs from conventional voxel-based methods in that it does not perform vertical binning, which makes hyperparameter tuning easier and allows for efficient handling of sparse data structures. Second, the Backbone network takes the generated pseudo-image as input and performs high-level feature extraction using 2D CNN. This backbone consists of down-sampling blocks that reduce the resolution of feature maps and blocks that up-sample and combine them to match the original resolution. By integrating features with different Receptive Fields, it acquires representations capable of handling objects of various sizes. Third, the Single Shot Detector (SSD), widely used in object detection, is adopted for the Detection Head. Based on the output feature maps from the backbone, it performs regression of 3D bounding boxes (position, size, angle) and class classification.

In evaluation experiments, PointPillars showed performance surpassing conventional methods boasting the highest accuracy (such as MV3D, VoxelNet, SECOND) on the KITTI 3D object detection benchmark (in both BEV and 3D detection). Notable is its inference speed, achieving extremely fast operation of 62 Hz on a desktop GPU. This is a result of eliminating 3D convolutions and using only optimized 2D convolution operations. Furthermore, it has been demonstrated that by using a Learned Encoder instead of the fixed encoders (manually designed features) used in conventional methods, significant accuracy improvement is achieved without sacrificing speed.

Although PointPillars proposed by Lang et al. is an excellent method in terms of the balance between inference speed and detection accuracy, several technical issues due to its structural characteristics have also become clear. First, false detections tend to occur in the identification of objects with similar geometric features. Since PointPillars compresses point clouds into vertical pillars for feature extraction, detailed information in the height direction is abstracted. Consequently, cases where long, thin vertical structures like poles or tree trunks are misidentified as pedestrians, or where pedestrians and cyclists are misclassified as each other, have been reported. Also, in the detection of passenger cars, cases of misdetecting similarly shaped vehicle classes like vans or trams (False Positives) have been confirmed.Next is the limitation of detection ability under poor observation conditions. Similar to other LiDAR-based methods, there is a tendency for missed detections (False Negatives) to occur when the target is partially occluded by other objects or is far from the sensor with extremely low point cloud density. Furthermore, the trade-off between spatial resolution and processing speed is also an issue. Ablation studies have shown that if the grid size is set large to prioritize inference speed, the detection performance for large objects like vehicles is maintained, but the detection accuracy for small targets like pedestrians and cyclists decreases. Therefore, to detect diverse targets simultaneously and at high speed in real environments, this trade-off needs to be carefully adjusted.



\section{ReID3D}

Guo et al. proposed a new framework "ReID3D" using only LiDAR sensors to overcome the problem of environmental dependency in conventional camera-based methods for the Person Re-identification (ReID) task, which plays an important role in public security and surveillance systems. Conventional ReID methods using RGB cameras have a fundamental issue of strongly depending on appearance information such as clothing color and texture. Therefore, visual information is missing at night or in low-light environments, and visual ambiguity increases in complex backgrounds, leading to a significant drop in identification accuracy. To address this problem, studies using Kinect or millimeter-wave radar as approaches utilizing depth information exist. However, Kinect is limited to indoor use with a narrow measurement range, and radar has low resolution, making detailed person identification difficult.

To solve these issues, Guo et al. focused on LiDAR, which is unaffected by lighting conditions and can acquire high-precision 3D structural information even at long distances. They aimed to realize ReID robust to clothing changes and darkness by extracting intrinsic features such as height, body shape, and gait from LiDAR point clouds. However, there was no precedent for research on ReID using LiDAR, and the lack of datasets necessary for learning was a major barrier. To solve this data shortage, the authors constructed the first real-world LiDAR ReID dataset "LReID" and a simulation dataset "LReID-sync." LReID is real data containing 320 IDs collected in diverse outdoor environments, while LReID-sync is synthetic data of 600 individuals generated using Unity3D.

The core of this method is a "Multi-task Pre-training" strategy to compensate for the sparsity of point cloud data and the lack of information due to a single viewpoint. First, when training the encoder using LReID-sync, two tasks are imposed: "Point Cloud Completion," which restores the complete overall shape from missing point clouds, and "SMPL parameter learning," which estimates the parameters of a human body model. This allows the encoder to acquire the geometric structure of the human body in advance, effectively supporting learning on real data. Also, a newly designed "Graph-based Complementary Enhancement Encoder (GCEE)" was adopted for the ReID network. GCEE uses a GCN that dynamically constructs graphs in the feature space as a backbone and introduces a Complementary Feature Extractor (CFE) with an "Eraser" module. The CFE eliminates information duplication between frames by erasing major features extracted from one frame before processing the next, enabling more comprehensive and highly discriminative feature extraction. Furthermore, a Transformer is used to integrate time-series information, efficiently aggregating dynamic features such as walking movements.

In evaluation experiments, ReID3D showed overwhelming robustness on the constructed LReID dataset compared to state-of-the-art camera-based methods (such as TCLNet). Especially in low-light environments where the accuracy of camera-based methods dropped significantly, ReID3D achieved a high performance of 93.3\% in Rank-1 accuracy (94.0\% overall), demonstrating the superiority of LiDAR which is independent of lighting conditions. In conclusion, ReID3D is a pioneering study that succeeded in extracting intrinsic 3D features of individuals from LiDAR point clouds by combining pre-training utilizing simulation data and GCEE specialized for point cloud characteristics, solving the challenge of person re-identification under adverse conditions which was difficult with conventional cameras alone.

While ReID3D shows excellent performance in low-light environments and in acquiring geometric structures, a challenge has been confirmed where its recognition accuracy is inferior to state-of-the-art video-based methods in bright (Normal light) environments with sufficient illumination. This is because video-based methods can maximize the use of rich appearance information (clothing color, pattern, texture, etc.) obtained in bright environments, whereas ReID3D, being a LiDAR-based method, relies only on shape information and reflection intensity and cannot use color information. Also, in the visualization analysis of the feature space, cases have been reported where ReID3D failed to identify specific pedestrians that would be easily identifiable by camera-based methods. These results suggest that LiDAR and cameras possess different modality characteristics and are in a complementary relationship with each other.



\section{Issues in Conventional Research and Positioning of This Study}

In person following by autonomous mobile robots, robustness when occlusion occurs remains an important issue to be solved. Methods using 2D LiDAR have low computational costs, but because the acquired information is limited to horizontal cross-sections, re-identification after occlusion or autonomous recovery from a lost state is difficult. On the other hand, methods using 3D LiDAR can acquire detailed spatial information, but most existing studies focus on the continuity of tracking, and methods for re-identification or recovery after tracking is completely interrupted by occlusion have not been sufficiently established. Additionally, while monocular cameras excel at re-identification using appearance information, they have issues such as the target moving out of the frame due to limited field of view and vulnerability to changes in lighting conditions.

Therefore, in this study, we propose a person following system using a robotic omnidirectional 3D LiDAR, "Livox Mid-360." By preventing the target from moving out of the frame with an omnidirectional field of view and utilizing point cloud data for re-detection and re-identification after occlusion, we aim to construct a system capable of autonomously continuing tracking even in complex environments.

\chapter{Human-following System Using 3D LiDAR and ReID3D}
\section{System Overview}

In this study, we propose a system that incorporates a LiDAR-based person re-identification model, "ReID3D," to autonomously recover from a "lost state" after an occlusion occurs during person following using an omnidirectional 3D LiDAR. This system is mainly composed of three phases: "pedestrian detection," "registration and tracking of the target," and "re-identification after occlusion."

First, we adopt "PointPillars," which is a 3D point cloud object detection network, as a detection model to identify the positions of pedestrians in the environment. The publicly available pre-trained model of PointPillars uses the KITTI dataset \cite{KITTI} collected with a rotating LiDAR (Velodyne HDL-64E). However, the scan pattern and point cloud density are significantly different from the prism scanning type LiDAR, "Livox Mid-360," used in this study. The domain gap caused by this difference in sensor methods leads to a decrease in detection accuracy. Therefore, in this study, we create an originally constructed pedestrian dataset using the Livox Mid-360 and perform re-training (fine-tuning) of PointPillars. By doing so, we construct a detector suitable for this system.

The specific processing flow of the system is as follows. First, surrounding pedestrians are detected using the re-trained PointPillars. At the start of tracking, features are extracted from the point cloud of the specified target using ReID3D, and the person is registered as the tracking target. During normal operation, the robot follows the target through tracking based on the detected position. If the tracking is interrupted due to occlusion by obstacles during following, the system transitions to the re-identification phase. In this phase, the point clouds of all pedestrians detected within the field of view are input into ReID3D, and they are compared and matched with the features of the target registered in advance. Thereby, the system correctly re-identifies the tracking target from the environment after occlusion and realizes the autonomous resumption of the following operation.



\section{System Configuration}

Figure. \ref{System overview.} shows the overall configuration of the person following system proposed in this study. This system is built using Robot Operating System 2 (ROS 2) as middleware. For the pedestrian detection model, we adopted ros2\_tao\_pointpillars, which is a ROS 2 implementation of PointPillars. Although Figure. \ref{System overview.} mainly illustrates the processing flow during the execution of following, the operation of this system differs significantly between the "target person registration phase" immediately after the start of following and the subsequent "following phase." Therefore, the details will be described below by dividing them into these two phases.

\begin{figure*}[h]
    \begin{center}
    \includegraphics[height=100mm,clip]{figs/System_overview.png}
    \caption{System overview.}
    \label{System overview.}
    \end{center}
\end{figure*}

\paragraph{Registration Phase} In this phase, first, the point cloud data acquired from the 3D LiDAR is input into ros2\_tao\_pointpillars to detect surrounding pedestrians. Generally, in the initialization of the tracking target, a method of selecting the detection result closest to the robot is used. However, due to false positives of the detector, there is a possibility that nearby walls or obstacles are mistakenly detected as people. To prevent such erroneous registration of non-human objects, this system sets spatial constraints on the search range. Specifically, a rectangular area with a width of 1.0 m and a depth of 3.0 m is set in the robot coordinate system. Among the detection results existing within this area, the person closest to the robot is selected as the tracking target. After the target is determined, the system starts tracking that person. At the same time, it extracts features from the target's point cloud using ReID3D and registers them as the tracking target.

\paragraph{Following Phase} In this phase, point cloud data acquired from the 3D LiDAR is input into ros2\_tao\_pointpillars, and pedestrian detection is performed continuously. Detected pedestrians are tracked by a tracking algorithm, and association with the tracking target is performed based on information up to the previous frame. Based on this tracking result, the system generates motion control commands for the robot and executes following of the target. On the other hand, if the tracking of the target is interrupted for 1.0 seconds due to occlusion or other reasons, the system determines that the target is lost (lost state) and transitions to the re-identification process. In the re-identification process, the point clouds of all pedestrians detected within the field of view at that time are input into ReID3D to extract features. By comparing and matching the obtained features of each pedestrian with the features of the tracking target saved in advance during the registration phase, the tracking target is re-identified.



\section{Software Configuration}

The software stack of the laptop used for the robot is shown in Figure. \ref{Software stack.}. The laptop is equipped with an NVIDIA RTX 3070 GPU (8GB memory), and we used Ubuntu 22.04 LTS as the operating system.

In implementing the deep learning models, PointPillars and ReID3D, we constructed independent Docker container environments for each model. This approach was taken to resolve dependency issues. Specifically, the two models require different versions of the CUDA toolkit and drivers, which makes it difficult for them to coexist within a single host environment. By separating them into containers, we solved this problem.

We adopted Robot Operating System 2 (ROS 2) Foxy Fitzroy as the middleware. Each Docker container and the host PC communicate and coordinate with each other via the ROS 2 communication protocol. Furthermore, we developed a custom ROS 2 package named "HARRP (Human-following Autonomous Robot system with ReID3D and PointPillars)." This package was designed to integratedly manage the entire processing pipeline, ranging from detection by PointPillars and identification by ReID3D to the final control of the robot. By simply launching this HARRP package, the system can be executed as a single integrated system. This allows the user to operate it without needing to be aware of the underlying distributed container environments.

\begin{figure*}[h]
    \begin{center}
    \includegraphics[height=120mm,clip]{figs/Software_stack.png}
    \caption{Software stack.}
    \label{Software stack.}
    \end{center}
\end{figure*}

\section{Dataset Construction}
\subsection{Data Collection}
The data collection environment is shown in Figure. \ref{Data collection environment.}. The dimensions of the laboratory used for the experiment are $6.0\,\mathrm{m} \times 7.1\,\mathrm{m}$. The LiDAR sensor was fixed to a dedicated stand as shown in Figure. \ref{LiDAR and custom stand.} and installed in the center of the room.

In this experiment, we specifically designed and fabricated the custom stand shown in Figure. \ref{LiDAR and custom stand.} to simulate the conditions where the sensor is mounted on an actual robot. This stand was designed to match the sensor mounting position and configuration of the robot shown in Figure. \ref{Height comparison.}. Accordingly, the mounting height of the LiDAR was set to be $119.5\,\mathrm{mm}$ from the ground.

\clearpage

\begin{figure*}[h]
    \begin{center}
    \includegraphics[height=70mm,clip]{figs/data_collection_room.JPG}
    \caption{Data collection environment.}
    \label{Data collection environment.}
    \end{center}
\end{figure*}

\begin{figure*}[h]
  \begin{minipage}[b]{0.5\linewidth}
    \centering
    \includegraphics[keepaspectratio, width=5cm, angle=-90]{figs/lidar.JPG}
    \caption{LiDAR and custom stand.}
    \label{LiDAR and custom stand.}
  \end{minipage}
  \begin{minipage}[b]{0.5\linewidth}
    \centering
    \includegraphics[keepaspectratio, height=5cm]{figs/kachaka_lidar.JPG}
    \caption{Height comparison.}
    \label{Height comparison.}
  \end{minipage}
\end{figure*}

Five subjects participated in the data collection. The appearances of the subjects are shown in Figure \ref{Overview of subjects captured for data collection (1/2).} and Figure \ref{Overview of subjects captured for data collection (2/2).}. To ensure visual diversity of pedestrians in the dataset, the subjects wore various types of clothing suitable for different seasons and attributes, such as T-shirts, long-sleeve shirts, hoodies, and work uniforms. Furthermore, to reproduce diverse pedestrian appearances in real-world environments, measurements were taken with various combinations of daily items. These included not only clothing but also accessories such as backpacks, suitcases, tote bags, and sunglasses.

The subjects walked around the LiDAR following five walking patterns shown in Figure \ref{Walking patterns in data collection.}, and a total of 12 data sets were collected. From each set, 85 frames were randomly extracted, resulting in a total of 1,020 frames of training data. Additionally, for the validation dataset, data were collected from three different subjects in a different environment, and 200 frames were randomly extracted in the same manner. Combining these, a dataset consisting of a total of 1,220 frames was constructed.

%===========================================================
% 1ページ目:前半 20枚
%===========================================================
\begin{figure}[p]
    \centering
    % リスト内の改行コードがスペース化しないよう % を付加し、カンマ後のスペースも削除
    \foreach \n [count=\i] in {%
        1,2,3,4,5,6,7,8,9,10,11,12,13,14,15,16,17,18,19,20%
    }{
        \subfloat[]{
            % ↓ ここも念のため確認: \n.JPG の間にスペースがないか
            \includegraphics[width=0.17\linewidth]{figs/\n.png}
        }%
        \ifnum\numexpr\i-((\i-1)/4)*4\relax=0 \par\vspace{1.0mm} \else \hfill \fi
    }
    \caption{Overview of subjects captured for data collection (1/2).}
    \label{Overview of subjects captured for data collection (1/2).}
\end{figure}

%===========================================================
% 2ページ目:後半 13枚
%===========================================================
\begin{figure}[p]
    \centering
    % リスト内の改行コードがスペース化しないよう % を付加し、カンマ後のスペースも削除
    \foreach \n [count=\i] in {%
        21,22,23,24,25,26,27,28,29,30,31,32,33%
    }{
        \subfloat[]{
            % ↓ ここも念のため確認: \n.JPG の間にスペースがないか
            \includegraphics[width=0.2\linewidth]{figs/\n.png}
        }%
        \ifnum\numexpr\i-((\i-1)/4)*4\relax=0 \par\vspace{1mm} \else \hfill \fi
    }
    \caption{Overview of subjects captured for data collection (2/2).}
    \label{Overview of subjects captured for data collection (2/2).}
\end{figure}

\begin{figure}[htbp]
    \centering
    % --- 1段目(3枚) ---
    \subfloat[Straight trajectory.]{
        \includegraphics[width=0.4\linewidth]{figs/traj_straight.png}
    }
    \hfill
    \subfloat[Walking around the sensor.]{
        \includegraphics[width=0.4\linewidth]{figs/traj_curve.png}
    }

    \vspace{2mm} % 上下の段の間隔調整

    \subfloat[Circular trajectory.]{
        \includegraphics[width=0.4\linewidth]{figs/traj_circle.png}
    }
    \hfill
    % --- 2段目(2枚) ---
    \subfloat[Lateral meandering trajectory.]{
        \includegraphics[width=0.4\linewidth]{figs/traj_side.png}
    }

    \vspace{2mm}

    \subfloat[Longitudinal meandering trajectory.]{
        \includegraphics[width=0.4\linewidth]{figs/traj_updown.png}
    }
    
    \caption{Walking patterns in data collection.}
    \label{Walking patterns in data collection.}
\end{figure}



\subsection{Annotation}
The annotation process is shown in Figure. \ref{Annotation using BAT 3D.}. In this study, we used only point cloud data for the annotation task. We adopted BAT 3D as the annotation tool. BAT 3D is the successor software to 3D BAT \cite{3D BAT}, which is an open-source tool that runs on a web browser. We assigned bounding boxes for pedestrians to the collected point cloud data.

In BAT 3D, when a bounding box is placed, three views from the X, Y, and Z axes are displayed on the left side of the screen, as shown in Figure. \ref{Annotation using BAT 3D.}. On the right side of the screen, an interface for adjusting the position and posture is displayed. This feature facilitates fine adjustments in the 3D space. As a result, it became possible to perform accurate annotation of pedestrians.

\begin{figure*}[h]
    \begin{center}
    \includegraphics[width=150mm,clip]{figs/bat-3d.png}
    \caption{Annotation using BAT 3D.}
    \label{Annotation using BAT 3D.}
    \end{center}
\end{figure*}



\subsection{Data Augmentation}
Figure. \ref{Examples of data augmentation.} shows examples of data augmentation. In this study, we performed data augmentation on the point cloud data to improve the diversity of the training data. For this purpose, we used a program that we implemented independently. We mainly adopted the following three methods for the augmentation.

First is Ground Truth Sampling (GT-Sampling). This is a method that randomly extracts pedestrian point clouds contained in the training dataset and adds them to the target frame. Second is Local Transformation. In this method, we apply rotation, scaling, position noise, and point noise to each pedestrian point cloud individually. Third is Global Transformation. We apply rotation, scaling, and horizontal flipping to the entire point cloud in the frame collectively.

Table \ref{Data augmentation parameters.} shows the specific parameter settings for each method. Through these processes, we generated 15,000 frames from the 1,020 training frames and 4,500 frames from the 200 validation frames. In total, we constructed a dataset of 19,500 frames.

\begin{figure}[htbp]
    \centering
    \subfloat[Original point cloud.]{
        \includegraphics[width=0.48\linewidth]{figs/PointCloud_origin.png}
    }
    \hfill
    \subfloat[Augmented point cloud.]{
        \includegraphics[width=0.48\linewidth]{figs/PointCloud_aug.png}
    }
    \hfill
    \caption{Examples of data augmentation.}
    \label{Examples of data augmentation.}
\end{figure}

\begin{table}[h]
  \centering
  \caption{Data augmentation parameters.}
  \label{Data augmentation parameters.}
  \begin{tabular}{lr} \toprule
    Parameter & Value \\ \midrule
    
    % GT-Sampling Settings
    \multicolumn{2}{l}{\textbf{GT-Sampling Settings}} \\ 
    Maximum number of added persons & $20$ \\
    $x$-coordinate range [m] & $-6$ to $10$ \\
    $y$-coordinate range [m] & $-6$ to $6$ \\ \midrule

    % Local Transformation Settings
    \multicolumn{2}{l}{\textbf{Local Transformation Settings}} \\
    Rotation around $z$-axis [rad] & $-\pi/2$ to $\pi/2$ \\
    Rotation around $x$- and $y$-axes [rad] & $-\pi/36$ to $\pi/36$ \\
    Scale  & $0.95$ to $1.05$ \\
    Position noise std. dev. [m] & $0.1$ \\
    Point clouds noise std. dev. [m] & $0.01$ \\ \midrule

    % Global Transformation Settings
    \multicolumn{2}{l}{\textbf{Global Transformation Settings}} \\
    Global rotation [rad] & $-\pi/4$ to $\pi/4$ \\
    Global scale & $0.95$ to $1.05$ \\
    Flip probability & $0.5$ \\ \bottomrule
  \end{tabular}
\end{table}



\section{Training of PointPillars}
We used the dataset constructed in the previous section to train the PointPillars model. The training was conducted using the NVIDIA TAO Toolkit. The TAO Toolkit is a framework that supports efficient training and optimization of deep learning models. It also provides pre-trained models for PointPillars.

In this study, we used the PointPillars training pipeline provided by the TAO Toolkit. We performed fine-tuning using our constructed dataset. For the training hardware, we used a desktop PC equipped with two NVIDIA RTX 3090 (24GB) GPUs. The hyperparameters used for training are shown in Table \ref{Hyper parameters for PointPillars training.}.

This section describes the hyperparameter settings used for training this model. The training was conducted for 800 epochs, and the batch size per GPU was set to 8. We adopted the Adam optimizer. For learning rate scheduling, we applied the One Cycle Policy.

The One Cycle Policy is a method that first increases the learning rate from an initial value to a maximum value as training progresses, and then decreases it again. This approach helps to achieve both faster training and the suppression of overfitting. The base learning rate was set to 0.003. We allocated 40\% of the total training steps (pct\_start = 0.4) to the phase where the learning rate increases. In the remaining period, the learning rate decreases.

In addition, we adjusted the momentum in the range of 0.95 to 0.85. This change is inversely correlated with the learning rate. This aims to improve the efficiency of parameter updates.

Furthermore, we introduced gradient clipping (Gradient Norm Clip) to ensure the stability of the training. This process limits the magnitude of the gradient when its L2 norm, calculated by backpropagation, exceeds a certain threshold (10.0 in this experiment). This prevents sudden changes in parameters, such as gradient explosion, and achieves stable training.

\begin{table}[h]
  \centering
  \caption{Hyper parameters for PointPillars training.}
  \label{Hyper parameters for PointPillars training.}
  \begin{tabular}{lr} \toprule
    Parameter & Value \\ \midrule
    batch\_size\_per\_gpu & 8 \\
    num\_epochs & 800 \\
    optimizer & adam\_onecycle \\
    lr & 0.003 \\
    weight\_decay & 0.01 \\
    momentum & 0.9 \\
    moms & [0.95, 0.85] \\
    pct\_start & 0.4 \\
    div\_factor & 10.0 \\
    decay\_step\_list & [35, 45] \\
    lr\_decay & 0.1 \\
    lr\_clip & 0.0000001 \\
    lr\_warmup & False \\
    warmup\_epoch & 1 \\
    grad\_norm\_clip & 10.0 \\ \bottomrule
  \end{tabular}
\end{table}



\section{Tracking}
For the tracking process, we managed the object's 3-dimensional position $(x, y, z)$ and its respective velocities $(v_x, v_y, v_z)$ as state variables. To estimate the true position from the point cloud data, which contains observation noise, we implemented a linear Kalman Filter based on a constant velocity model.

This algorithm consists of two stages: the Prediction Step'' and the Update Step.'' In the Prediction Step, the system predicts the state at the next time step based on a physical model. In the Update Step, the system corrects the estimated values using the observation values obtained from the sensors.

\subsection{Definition of State Variables and Model}
The state vector of the system, $\mathbf{x}_k$, is defined as the following 6-dimensional vector. It consists of position and velocity components.

\begin{equation}
\mathbf{x}_k = \begin{bmatrix} x & y & z & v_x & v_y & v_z \end{bmatrix}^T_k
\end{equation}

We assume that the object moves at a constant velocity during a small time interval $dt$. Based on this constant velocity model, the State Transition Matrix $\mathbf{F}$ is defined as follows. Note that in our implementation, the value of $dt$ is dynamically updated as the elapsed time from the previous frame.

\begin{equation}
\mathbf{F} = \begin{bmatrix} 
1 & 0 & 0 & dt & 0 & 0 \\
0 & 1 & 0 & 0 & dt & 0 \\
0 & 0 & 1 & 0 & 0 & dt \\
0 & 0 & 0 & 1 & 0 & 0 \\
0 & 0 & 0 & 0 & 1 & 0 \\
0 & 0 & 0 & 0 & 0 & 1 
\end{bmatrix}
\end{equation}

\subsection{Prediction Step}
In this step, we predict the state at the current time $k$ from the estimated value at the previous time $k-1$.

\begin{itemize}
    \item \textbf{State Prediction}:
    We linearly predict the next position based on the current estimated velocity.
    \begin{equation}
    \mathbf{\hat{x}}_{k|k-1} = \mathbf{F} \mathbf{\hat{x}}_{k-1|k-1}
    \end{equation}
    
    \item \textbf{Covariance Prediction}:
    We update the error covariance matrix $\mathbf{P}$. This matrix represents the uncertainty of the estimation. Here, $\mathbf{Q}$ is the process noise covariance matrix. It represents the allowed deviation from the constant velocity model (such as changes in acceleration). In this implementation, we set the diagonal elements to $0.1$.
    \begin{equation}
    \mathbf{P}_{k|k-1} = \mathbf{F} \mathbf{P}_{k-1|k-1} \mathbf{F}^T + \mathbf{Q}
    \end{equation}
\end{itemize}

\subsection{Update Step}
In this step, we correct the predicted values using the observation values $\mathbf{z}_k$ obtained from PointPillars. The observation vector is a 3-dimensional vector consisting only of position: $\mathbf{z}_k = [z_x, z_y, z_z]^T$.

\begin{itemize}
    \item \textbf{Observation Model}:
    We use the observation matrix $\mathbf{H}$ to extract the observable 3-dimensional position components from the 6-dimensional state vector.
    \begin{equation}
    \mathbf{H} = \begin{bmatrix} 
    1 & 0 & 0 & 0 & 0 & 0 \\ 
    0 & 1 & 0 & 0 & 0 & 0 \\ 
    0 & 0 & 1 & 0 & 0 & 0 
    \end{bmatrix}
    \end{equation}

    \item \textbf{Calculation of Kalman Gain}:
    We calculate the Kalman Gain $\mathbf{K}_k$. This gain determines the balance of reliability between the predicted value and the observed value. Here, $\mathbf{R}$ is the observation noise covariance matrix. It represents the measurement error of the sensor. In this implementation, we set $\mathbf{R} = 1.0 \cdot \mathbf{I}$.
    \begin{align}
    \mathbf{S}_k &= \mathbf{H} \mathbf{P}_{k|k-1} \mathbf{H}^T + \mathbf{R} \\
    \mathbf{K}_k &= \mathbf{P}_{k|k-1} \mathbf{H}^T \mathbf{S}_k^{-1}
    \end{align}

    \item \textbf{State Update}:
    We calculate the observation residual $\mathbf{y}_k = \mathbf{z}_k - \mathbf{H}\mathbf{\hat{x}}_{k|k-1}$. Then, we multiply this residual by the Kalman Gain to correct the predicted value. This gives us the final estimated value.
    \begin{equation}
    \mathbf{\hat{x}}_{k|k} = \mathbf{\hat{x}}_{k|k-1} + \mathbf{K}_k (\mathbf{z}_k - \mathbf{H}\mathbf{\hat{x}}_{k|k-1})
    \end{equation}

    \item \textbf{Covariance Update}:
    We reduce the uncertainty of the estimation as the observation information is incorporated.
    \begin{equation}
    \mathbf{P}_{k|k} = (\mathbf{I} - \mathbf{K}_k \mathbf{H}) \mathbf{P}_{k|k-1}
    \end{equation}
\end{itemize}

\subsection{Implementation Features}
In this implementation, we address the issue of asynchronous topic reception in the ROS2 environment. Therefore, we do not use a fixed time step. Instead, we dynamically calculate the State Transition Matrix $\mathbf{F}$ using the elapsed time $dt$ from the previous update.

In addition, at the time of initialization, we set the error covariance matrix $\mathbf{P}$ to a large value, specifically $10.0 \cdot \mathbf{I}$. By doing this, the system is designed to place more importance on the observation values immediately after the tracking starts. This ensures that the estimated value converges rapidly to the true value.



\chapter{Experiments}
\section{Experimental Objectives}
The purpose of this study is to develop a human-following system using a 3D LiDAR. The main goal is to create a system that can re-identify the target person and continue tracking them even after they are hidden by obstacles (occlusion).

To verify the effectiveness of this system, we adopted a 3D LiDAR made by Livox as the sensor in our experiments. This sensor is designed for robots and has a unique scanning pattern. Therefore, we use PointPillars, which is an object detection network optimized for the characteristics of the point cloud generated by this sensor. We will quantitatively evaluate the accuracy of pedestrian detection using this setup.

In addition, we use a public dataset to evaluate the performance of re-identification. Specifically, we test whether the developed system can correctly identify the target person again after an occlusion occurs. Through these experiments, we aim to clarify the overall performance of the process, ranging from pedestrian detection to re-identification using 3D LiDAR. Finally, we demonstrate the usefulness of the human-following system we constructed.



\section{Experimental Methods}
In order to verify the effectiveness of the proposed system, we conduct two types of experiments.

The first experiment is a quantitative evaluation of the pedestrian detection accuracy of PointPillars. For this evaluation, the model has been optimized for the characteristics of the point cloud generated by the Livox 3D LiDAR. The second experiment is an evaluation of the tracking performance and re-identification performance of the entire human-following system that we constructed.

In addition, for the experimental equipment, we used a laptop PC equipped with an NVIDIA RTX 3070 (8GB) GPU.

\subsection{Comparative Experiment on Pedestrian Detection Performance}

In this experiment, we evaluate the pedestrian detection performance of PointPillars using the 4,500-frame validation dataset constructed in Section 3.4.

As evaluation metrics, we adopted BEV (Bird’s Eye View) AP and 3D AP, which are widely used in object detection. BEV AP is a metric calculated based on the area overlap with the ground truth, obtained by projecting the estimated 3D bounding box onto a 2D plane (bird's eye view). On the other hand, 3D AP is calculated based on the overlap of the 3D position and volume (IoU: Intersection over Union) of the bounding box. In this experiment, the IoU threshold was set to 0.5, following general standards for pedestrian detection.

Regarding the experimental setup for inference, we adopted a format where input data is converted from binary point cloud files and published as ROS 2 topics, which are then subscribed to by the detection model. The publishing frequency of the topics was set to 10 Hz. Furthermore, to verify the effectiveness of our method, we used a model available in the ros2\_tao\_pointpillars package, which is an existing open-source implementation, as a comparison target.

\subsection{Comparative Experiment on Tracking and Re-identification Performance}

In this experiment, we evaluate the tracking and re-identification performance of the entire person-following system we constructed, using a public dataset. For the evaluation dataset, we selected TPT-BENCH \cite{TPT-Bench}. This is a large-scale dataset that specializes in the robot Target Person Tracking (TPT) task and includes 3D LiDAR data, which is crucial for this study. Compared to existing datasets such as RoboSense \cite{RobSense} and JRDB-PanoTrack \cite{JRDB-PanoTrack}, TPT-BENCH includes long-term tracking in diverse environments (indoor/outdoor, lighting changes, and crowded environments). Therefore, it is optimal for evaluating re-identification ability.

In this study, we selected "Sequence 0015" from the 48 available sequences as the subject of evaluation. The specifications of this sequence are as follows. The total duration where Ground Truth exists is 202.1 seconds. Within this, the total disappearance time, during which the target becomes untrackable, is 24.3 seconds, and the number of disappearances is recorded as 28 times. Regarding environmental characteristics, although the target's clothing does not change within the sequence, the lighting conditions tend to shift from standard brightness to a dim state.

Details regarding the comparison methods and evaluation metrics used in this experiment are described in Section \ref{baseline} and Section \ref{evaluation}, respectively.

\subsubsection{Comparison Methods} \label{baseline} We selected the following methods for comparison in this experiment.

\paragraph{SiamRPN++ \cite{SiamRPN++}} This is a representative method for Visual Object Tracking (VOT) using Siamese networks. It solves the problem of position invariance caused by padding in deep networks (such as ResNet) by using a spatial sampling strategy. We adopt this as a baseline for image-based Single Object Tracking (SOT) methods that do not use depth information such as LiDAR.

\paragraph{DiMP \cite{DiMP}} This is a method that optimizes a model to distinguish between the target and the background online during inference. Addressing the issue that conventional Siamese-based methods cannot utilize background information, it achieves high discrimination ability by incorporating an optimization process that minimizes discriminative loss directly into the network. We use this for comparison as a representative of SOT methods that use a discriminative approach.

\paragraph{MixFormer \cite{MixFormer}} This is a tracking framework based on Transformers, which has been gaining attention in recent years. By using the Mixed Attention Module (MAM), it performs feature extraction and the integration of target information simultaneously. This allows the model to learn the dense interaction between the target and the search area. We adopt this to confirm the performance of Transformer-based SOT methods.

\paragraph{OCL-RPF \cite{OCL-RPF}} This is a tracking method based on Person Re-identification (ReID), specialized for Robot Person Following (RPF). To address the challenge that conventional fixed models cannot adapt to environmental changes, it has a mechanism to dynamically update the feature extractor using Online Continual Learning (OCL). We use this for comparison as a method that aims for "long-term tracking from a robot perspective," similar to this study.

\subsubsection{Evaluation Metrics} \label{evaluation}
In this experiment, we use Average Overlap (AO), F-score, and Avg Max Recall (AMR), which are adopted in TPT-BENCH \cite{TPT-Bench}, as quantitative evaluation metrics.

First, let $P_t$ be the predicted bounding box and $G_t$ be the ground truth in each frame $t$. We define their Intersection over Union as $\text{IoU}_t$.

\begin{equation}
 \text{IoU}_t = \frac{|P_t \cap G_t|}{|P_t \cup G_t|}
\end{equation}

The definitions of each metric are described below.

\paragraph{Average Overlap (AO)}
AO is defined as the average of $\text{IoU}_t$ over all frames where the target exists. Let $N$ be the total number of frames; AO is expressed as follows:

\begin{equation}
 \text{AO} = \frac{1}{N} \sum_{t=1}^{N} \text{IoU}_t
\end{equation}

\paragraph{F-score}
We use Tracking Precision ($Pr$) and Tracking Recall ($Re$) to calculate the F-score.
We define a prediction satisfying the IoU threshold as a True Positive, and let $N_{TP}$ be the number of such frames. Also, we define $N_{pred}$ as the total number of frames where the system output a prediction, and $N_{GT}$ as the total number of frames where the target actually exists. In this case, $Pr$ and $Re$ are expressed as follows:

\begin{equation}
 Pr = \frac{N_{TP}}{N_{pred}}, \quad Re = \frac{N_{TP}}{N_{GT}}
\end{equation}

The F-score is defined as the harmonic mean of $Pr$ and $Re$.

\begin{equation}
 \text{F-score} = \frac{2 \cdot Pr \cdot Re}{Pr + Re}
\end{equation}

$Pr$ represents the accuracy of the prediction (low false positives), and $Re$ represents the target detection rate (low missed detections). The F-score is a metric that evaluates these comprehensively.

\paragraph{Avg Max Recall (AMR)}
AMR is the average of the maximum $Re$ achievable under the condition that no False Positives are allowed ($Pr = 100\%$), calculated over multiple IoU thresholds. Let $\Omega$ be the set of IoU thresholds, and $Re_{max}(\lambda)$ be the maximum recall at a certain threshold $\lambda \in \Omega$. AMR is defined by the following equation:

\begin{equation}
 \text{AMR} = \frac{1}{|\Omega|} \sum_{\lambda \in \Omega} Re_{max}(\lambda) \quad \text{s.t.} \quad Pr = 1
\end{equation}

This metric evaluates the ability to maintain tracking without false tracking.

\clearpage

\section{Experimental Results}
\subsection{Comparative Experiment on Pedestrian Detection Performance}
The experimental results for pedestrian detection are shown in Table \ref{Evaluation results of pedestrian detection.}.
In contrast to ros2\_tao\_pointpillars, which achieved a BEV AP of 35.47\% and a 3D AP of 18.50\%, our method achieved a BEV AP of 79.86\% and a 3D AP of 61.59\%. This indicates that our method significantly outperformed the conventional model in both metrics.

\begin{table}[h]
  \centering
  \caption{Evaluation results of pedestrian detection.}
  \label{Evaluation results of pedestrian detection.}
  \begin{tabular}{ccc} \toprule
    Model & BEV AP (\%) & 3D AP (\%) \\ \midrule
    ros2\_tao\_pointpillars & 35.47 & 18.50 \\
    Ours & 79.86 & 61.59 \\ \bottomrule
  \end{tabular}
\end{table}

Figure \ref{training_curves} shows the evolution of loss functions and learning rate during this experiment. Here, in addition to the total loss (Figure \ref{train_loss}), we present its breakdown: bounding box classification loss (Figure \ref{loss_cls}), localization loss (Figure \ref{loss_loc}), and direction estimation loss (Figure \ref{loss_dir}).

First, a common feature observed in all graphs is a temporary spike in loss around Step 47,000. This behavior is attributed to resuming the training from the 501st epoch. However, apart from this local fluctuation, the overall loss shows a decreasing trend as training progresses, indicating that the training proceeded appropriately.

Looking at the detailed trends, for Figure \ref{train_loss}, \ref{loss_cls}, and \ref{loss_loc}, the loss drops rapidly from the beginning of training (Step 0) and continues to decrease steadily until around Step 700. Thereafter, although the rate of decrease becomes more gradual, a trend toward steady convergence was observed.

On the other hand, the Direction Loss (Figure \ref{loss_dir}) showed a different trend compared to the other losses. The decrease remained gradual until around Step 25,000, but the decreasing trend became stronger after that. The steady decrease continued even after resuming training, and from around Step 69,000, the rate of decrease slowed down, becoming almost flat (a state of convergence).

\clearpage

\begin{figure}[h]
    \centering
    % --- Row 1 ---
    \subfloat[Total training loss]{
        \includegraphics[width=0.48\linewidth]{figs/fig_train_loss.png}
        \label{train_loss}
    }
    \hfill % Adjust spacing automatically
    \subfloat[Classification loss]{
        \includegraphics[width=0.48\linewidth]{figs/fig_loss_cls.png}
        \label{loss_cls}
    } \\ % Line break
    \vspace{3mm} % Adjust vertical spacing

    % --- Row 2 ---
    \subfloat[Localization loss]{
        \includegraphics[width=0.48\linewidth]{figs/fig_loss_loc.png}
        \label{loss_loc}
    }
    \hfill
    \subfloat[Direction loss]{
        \includegraphics[width=0.48\linewidth]{figs/fig_loss_dir.png}
        \label{loss_dir}
    } \\

    \caption{Evolution of loss functions and learning rate.}
    \label{training_curves}
\end{figure}

Figure \ref{bbox} shows examples of pedestrian detection. Note that the validation data used here contains three pedestrians. First, regarding trends common to both models, although pedestrians were generally detected, we confirmed that missed detections (False Negatives) occurred in some scenes. Also, false detections (False Positives), where non-human objects such as chairs or walls were mistakenly detected as pedestrians, were occasionally observed.

Focusing on the characteristics of each model, in the conventional method \texttt{ros2\_tao\_pointpillars}, as shown in Figure \ref{eval_related}, we observed cases where two pedestrians close to each other could not be separated and were detected together as a single bounding box. On the other hand, in the proposed method, as shown in Figure \ref{eval_harrp}, individual pedestrians were correctly separated and detected, yielding good detection results overall. However, as a specific issue, we confirmed a phenomenon where detection failed when the pedestrian was stationary and standing upright.

\clearpage

\begin{figure}[h]
    \centering
    % --- Left Figure ---
    \subfloat[Detection results using the conventional model.]{
        \includegraphics[width=0.48\linewidth]{figs/eval_pp_related.png}
        \label{eval_related}
    }
    \hfill % Automatically adjusts space between figures
    % --- Right Figure ---
    \subfloat[Result of the optimized PointPillars (Ours).]{
        \includegraphics[width=0.48\linewidth]{figs/eval_pp_harrp.png}
        \label{eval_harrp}
    }
    
    \caption{Examples of pedestrian detection by the conventional model and Ours.}
    \label{bbox}
\end{figure}

\subsection{Quantitative Evaluation on the TPT-Bench Dataset}
The experimental results using the TPT-Bench dataset are shown in Table \ref{tpt-bench}. First, we discuss the trends of the overall metrics. In this experiment, the AMR was 0.00\% for all methods. Also, focusing on the relationship between AO (Average Overlap) and F-score, we confirmed a tendency that the F-score shows higher values than AO in all methods.

Next, we compare the performance of each method. Regarding SiamRPN++ and MixFormer, both AO and F-score were below 10\%, indicating that sufficient tracking performance was not obtained in Sequence 0015. In contrast, DiMP, OCL-RPF, and the proposed method all achieved scores of 15\% or higher, showing relatively good performance. Among them, DiMP achieved results exceeding 22\% in both AO and F-score. Notably, OCL-RPF recorded extremely high values exceeding 72\% in both metrics, showing an overwhelming performance difference compared to other comparison methods and the proposed method.

\clearpage

\begin{table}[h]
  \centering
  \caption{Evaluation results of tracking metrics (AO, F1-score, and AMR) on TPT-Bench.}
  \label{tpt-bench}
  \begin{tabular}{ccccc} \toprule
    Method & AO (\%) & F-score (\%) & AMR (\%) & Sensor \\ \midrule
    SiamRPN++ & 3.32 & 5.48 & 0.00 & Camera \\
    DiMP & 22.65 & 27.05 & 0.00 & Camera \\
    MixFormer & 3.93 & 5.41 & 0.00 & Camera \\
    OCL-RPF & 72.71 & 77.11 & 0.00 & Camera \\
    Ours & 15.87 & 16.12 & 0.00 & 3D LiDAR \\ \bottomrule
  \end{tabular}
\end{table}

Tracking results using our method on the TPT-Bench dataset are shown in Figure \ref{tracking} and Figure \ref{reid}.
In each figure, the left side shows the visualization of point cloud data from the 3D LiDAR, and the right side shows the camera image at the same timestamp.

In the 3D LiDAR point cloud images, red bounding boxes indicate the pedestrian detection results by PointPillars, and green bounding boxes indicate the target tracking results estimated by ReID3D and the Kalman filter. Above the tracking box, the tracking ID and the feature vector similarity (Sim) with the target, output by ReID3D, are displayed. Additionally, the gray spheres displayed around the detected pedestrians are markers indicating that the feature vector extraction process by ReID3D has been performed.

Figure \ref{tracking} shows the sequential tracking results in a situation where the target is hidden by obstacles in the environment and becomes temporarily unobservable. First, at the time of Figure \ref{tracking1}, the target is visible and is being tracked normally. Next, at the time of Figure \ref{tracking2}, the target has moved behind an obstacle, resulting in an occlusion state where direct detection by the LiDAR sensor is impossible. However, since this system performs state estimation using a Kalman filter, it is possible to predict the target's position even during periods when observation data is unavailable. As a result, as shown in Figure \ref{tracking3}, it was confirmed that the system prevented the target from being lost until re-appearance, achieving robust tracking.

Figure \ref{reid} shows an excerpt of the process of recovering from a temporary tracking interruption through re-identification using ReID3D. In the transition from Figure \ref{reid1} to Figure \ref{reid2} in this sequence, the system loses sight of the target due to a missed detection, and the tracking state shifts to "Lost." However, subsequently, as shown in Figure \ref{reid3}, even in a situation where multiple other pedestrians were present, re-identification by ReID3D functioned correctly. Consequently, the system successfully re-identified the target and resumed tracking.

\begin{figure}[h]
    \centering
    % --- 1st Image ---
    \subfloat[Tracking before occlusion.]{
        \includegraphics[width=1.0\linewidth]{figs/tracking1.png}
        \label{tracking1}
    } \\ % Line break
    \vspace{5mm}
    
    % --- 2nd Image ---
    \subfloat[Tracking during occlusion.]{
        \includegraphics[width=1.0\linewidth]{figs/tracking2.png}
        \label{tracking2}
    } \\ % Line break
    \vspace{5mm}

    % --- 3rd Image ---
    \subfloat[Tracking after occlusion.]{
        \includegraphics[width=1.0\linewidth]{figs/tracking3.png}
        \label{tracking3}
    }
    
    \caption{Sequential tracking results showing the target before, during, and after occlusion.}
    \label{tracking}
\end{figure}

\begin{figure}[h]
    \centering
    % --- 1st Image ---
    \subfloat[Target tracking in progress.]{
        \includegraphics[width=1.0\linewidth]{figs/reid1.png}
        \label{reid1}
    } \\ % Line break
    \vspace{5mm}
    
    % --- 2nd Image ---
    \subfloat[Target lost.]{
        \includegraphics[width=1.0\linewidth]{figs/reid2.png}
        \label{reid2}
    } \\ % Line break
    \vspace{5mm}
    
    % --- 3rd Image ---
    \subfloat[Target re-identified.]{
        \includegraphics[width=1.0\linewidth]{figs/reid3.png}
        \label{reid3}
    }
    
    \caption{Process of recovering the target from a lost state.}
    \label{reid}
\end{figure}

\clearpage

\section{Discussion}
\subsection{Comparative Experiment on Pedestrian Detection Performance}
From the experimental results, it was confirmed that PointPillars, which was optimized for the point cloud characteristics of the Livox 3D-LiDAR, demonstrated pedestrian detection performance that significantly exceeded that of the conventional model. This is considered to be because the training using the custom dataset proposed in this study enabled feature extraction adapted to the sensor characteristics, effectively overcoming the domain gap in pedestrian detection.

However, differences were observed between the two models regarding the tendency of false detections. In \texttt{ros2\_tao\_pointpillars}, cases were seen where pedestrians in close proximity were mistakenly detected as a single bounding box. This may be influenced by the fact that in the publicly available pre-trained model, annotations likely existed where two pedestrians were represented by a single bounding box. On the other hand, in our method, a phenomenon was observed where pedestrians were not detected when they were stationary and standing upright. This is thought to be because the dataset contained a large number of point clouds representing pedestrians in a moving state.

\subsection{Comparative Experiment on Tracking and Re-identification Performance}
The experimental results show that our method achieved tracking and re-identification performance surpassing the comparison targets, SiamRPN++ and MixFormer. SiamRPN++ basically adopts a Local Search approach that searches the vicinity of the estimated position from the previous frame, and MixFormer is a method that tracks by interacting the features of the target template and the search region. Since these methods rely on continuous visual information, their tracking performance tended to degrade significantly in environments where target disappearance occurs due to long-term occlusion, such as in Sequence 0015, or where visual features deteriorate due to lighting changes.

In contrast, our method adopts an approach that generates feature vectors of pedestrians from 3D LiDAR point clouds and performs re-identification (ReID). By utilizing point cloud information that is not affected by lighting conditions and performing detection-based global matching, our method demonstrated higher re-identification performance than the aforementioned comparison methods. As a result, tracking recovery after losing the target became possible, which is considered to have contributed to the improvement of the score in the quantitative evaluation.

\subsubsection{Impact of Online Adaptation Capability on Re-identification Performance}
The primary factor why the score of the proposed method did not reach the level of OCL-RPF lies in the difference in the target model update mechanism. OCL-RPF possesses an "Online Continual Learning" mechanism, which sequentially learns the target's appearance that changes from moment to moment during the tracking process (such as changes in posture or transitions in viewing angle) and dynamically adapts the model. This makes it possible to perform re-identification with high accuracy even if the target reappears with features different from the initial frame.

In contrast, our method performs ReID based on the similarity between the initial feature vector output from a pre-trained fixed model and the feature vector at the time of re-detection. Therefore, in re-identification after a loss, the system could not determine that the target was the same person, which likely caused a decisive difference in the success rate of tracking recovery.

\subsubsection{Limitations of Tracking Prediction by Kalman Filter and Comparison with DiMP}
Next, we discuss from the perspective of the tracking algorithm why our method fell short of DiMP (AO score: 15.87\% vs 22.65\%), which is a general-purpose tracker without a ReID function.

In our method, a Kalman filter is adopted for the dynamic tracking of the target, performing state estimation assuming a linear motion model such as constant velocity linear motion. However, the sequence of TPT-Bench used in this experiment includes non-linear behaviors due to the difference in movement between the pedestrian and the robot. In such scenes, a divergence (prediction error) tends to occur between the predicted position by the Kalman filter based on the linear model and the actual target position. Furthermore, the deviation in the predicted position becomes a factor that induces inconsistencies in the search region (Region of Interest) in the subsequent ReID step or errors in data association (ID assignment). As a result, even if the tracking itself continues, the overlap of the bounding box decreases due to prediction lag or overshoot, which is inferred to have led to the low AO score.

On the other hand, DiMP integrates IoU-Net. It does not solely rely on prediction by a motion model for target position estimation but performs regression to maximize the bounding box overlap (IoU) directly from image features. This allows for high-precision rectangle generation that "sticks" to the features in the image, even if the target shows irregular movements. In other words, the qualitative difference in tracking accuracy between our method's "motion prediction by Kalman filter" and DiMP's "learning-based rectangle regression" was reflected in the numerical results.

\subsubsection{Comprehensive Discussion on Target Disappearance and Recovery Capability}
Finally, we discuss the recovery capability throughout the experiment.
The background behind OCL-RPF (AO 72.71\%) recording an overwhelming score includes not only its adaptability through online continual learning but also its possession of a powerful global search capability that re-searches for the target from the entire image when the target is lost.

Our method is also equipped with a ReID function and is designed to perform a re-search by ReID when the target is not found within the prediction range of the Kalman filter. However, as mentioned above, due to the limits of the Kalman filter's prediction accuracy, if the occlusion continues for a long time and the target's motion deviates significantly from the prediction, the risk increases that the system cannot correctly narrow down the area to re-search or links the target to an incorrect cluster.

\chapter{Conclusion}
\section{Conclusion}
In this study, we developed a human-following system using 3D LiDAR. We verified its performance in pedestrian detection, tracking, and re-identification.

For this system, we adopted PointPillars, which was optimized for the point cloud characteristics of the Livox 3D LiDAR. By training the model with a custom dataset that we built independently, we successfully constructed a high-precision pedestrian detector. Furthermore, by integrating ReID3D, we realized a human-following system that can re-identify the target person even after occlusion (being hidden by obstacles) occurs.

To verify the effectiveness of the proposed method, we conducted evaluation experiments using both our custom dataset and a public dataset. The results showed that the optimized PointPillars significantly outperformed the conventional model in pedestrian detection. Thus, the usefulness of our method was confirmed.



\section{Future Work}
There are two main tasks for future work.

First is the further improvement of pedestrian detection performance. From the experimental results, we confirmed a tendency where the system failed to detect pedestrians when they stopped and stood upright. We believe the cause is that the dataset used for training was dominated by point clouds of moving pedestrians, and data of stationary pedestrians was insufficient. Therefore, we expect to improve the detection performance by constructing a new dataset that includes more point clouds of stopped pedestrians and performing retraining.

Second is the improvement of tracking and re-identification performance. Our system uses a linear Kalman Filter for tracking. However, there remains an issue where tracking becomes difficult when the relative trajectory becomes non-linear due to sudden movements of the pedestrian or the robot. To solve this, we believe that introducing non-linear estimation methods, such as the Extended Kalman Filter (EKF) or the Unscented Kalman Filter (UKF), would be effective.

In addition, ReID3D, which we used as the re-identification model, assumes scenarios like intersection monitoring. It relies on the assumption that the target is far away and the entire body is captured in the point cloud. However, in the human-following task, the distance between the LiDAR and the pedestrian is short. Consequently, situations frequently occur where only a part of the body is measured because the target goes out of the field of view. Therefore, it is necessary to develop a new model that can perform robust re-identification even from short-range and partial point cloud information.
\clearpage

% 以下の5行は修論謝辞用なのでコメントアウトしています。
%公聴会が終ってからLCに修論を提出する前にコメントアウトを解除してください。
%\kitchapter*{\KITifThesis{Acknowledgements}{Acknowledgements}}
%\thesisonly{\addcontentsline{toc}{chapter}{Acknowledgements}}
%\thesisonly{%
%\noindent
I would like to express my sincere gratitude to Prof.\ Kosei Demura for his continuous guidance,
insightful advice, and constructive feedback throughout this research.

I am also grateful to the members of the Demura Laboratory for their valuable discussions and
support in preparing the experimental environment and conducting evaluations.

Finally, I would like to thank my family for their encouragement and support during my graduate studies.

%}

% ----- Bibliography -----
% Use whichever workflow you prefer (BibTeX / BibLaTeX).
% For BibTeX:
% \bibliographystyle{plain}
\bibliography{refs}
\addcontentsline{toc}{chapter}{References}
\begin{thebibliography}{99}
\bibitem{ReID3D}
Guo W, Pan Z, Liang Y, et al. LiDAR-based person re-identification. In: Proceedings of the IEEE/CVF Conference on Computer Vision and Pattern Recognition (CVPR); 2024 Jun 17--21; Seattle, WA. Los Alamitos (CA): IEEE Computer Society; 2024. p. 17437--17447.

\bibitem{2D LiDAR}
Arras KO, Mozos OM, Burgard W. Using boosted features for the detection of people in 2D range data. In: Proceedings of the IEEE International Conference on Robotics and Automation (ICRA); 2007 Apr 10--14; Rome, Italy. Piscataway (NJ): IEEE; 2007. p. 3402--3407.

\bibitem{3D LiDAR}
Yan Z, Duckett T, Bellotto N. Online learning for 3D LiDAR-based human detection: experimental analysis of point cloud clustering and classification methods. Auton Robot. 2020;44:147--164.

\bibitem{Monocular}
Koide K, Miura J, Menegatti E. Monocular person tracking and identification with on-line deep feature selection for person following robots. Robot Auton Syst. 2020;124:103348.

\bibitem{Camera LiDAR}
Krebs S, Gavrila DM. Camera- and LiDAR-based person re-identification. In: Proceedings of the IEEE Intelligent Vehicles Symposium (IV); 2025 Jun 22--25; Cluj-Napoca, Romania. Piscataway (NJ): IEEE; 2025. p. 1408--1414.

\bibitem{PointPillars}
Lang AH, Vora S, Caesar H, et al. PointPillars: fast encoders for object detection from point clouds. In: Proceedings of the IEEE/CVF Conference on Computer Vision and Pattern Recognition (CVPR); 2019 Jun 16--20; Long Beach, CA. Los Alamitos (CA): IEEE Computer Society; 2019. p. 12697--12705.

\bibitem{KITTI}
Geiger A, Lenz P, Urtasun R. Are we ready for autonomous driving? The KITTI vision benchmark suite. In: Proceedings of the IEEE/CVF Conference on Computer Vision and Pattern Recognition (CVPR); 2012 Jun 16--21; Providence, RI. Los Alamitos (CA): IEEE Computer Society; 2012. p. 3354--3361.

\bibitem{3D BAT}
Zimmer W, Rangesh A, Trivedi M. 3D BAT: a semi-automatic, web-based 3D annotation toolbox for full-surround, multi-modal data streams. In: Proceedings of the IEEE Intelligent Vehicles Symposium (IV); 2019 Jun 9--12; Paris, France. Piscataway (NJ): IEEE; 2019. p. 583--590.

\bibitem{TPT-Bench}
Ye H, Zhan Y, Situ W, et al. TPT-Bench: a large-scale, long-term and robot-egocentric dataset for benchmarking target person tracking. arXiv preprint arXiv:2505.07446 [Internet]. 2025 [cited 2026 Jan 18]. Available from: https://arxiv.org/abs/2505.07446

\bibitem{JRDB-PanoTrack}
Martin-Martin R, Patel M, Rezatofighi H, et al. JRDB-PanoTrack: an open-world panoptic segmentation and tracking robotic dataset in crowded human environments. In: Proceedings of the IEEE/CVF Conference on Computer Vision and Pattern Recognition (CVPR); 2024 Jun 17--21; Seattle, WA. Los Alamitos (CA): IEEE Computer Society; 2024. p. 22325--22334.

\bibitem{RobSense}
Do MK, Han K, Lai P, et al. RobSense: a robust multi-modal foundation model for remote sensing with static, temporal, and incomplete data adaptability. In: Proceedings of the IEEE/CVF Conference on Computer Vision and Pattern Recognition (CVPR); 2025 Jun 11--15; Nashville, TN. Los Alamitos (CA): IEEE Computer Society; 2025. p. 7427--7436.

\bibitem{SiamRPN++}
Li B, Wu W, Wang Q, et al. SiamRPN++: evolution of Siamese visual tracking with very deep networks. In: Proceedings of the IEEE/CVF Conference on Computer Vision and Pattern Recognition (CVPR); 2019 Jun 16--20; Long Beach, CA. Los Alamitos (CA): IEEE Computer Society; 2019. p. 4282--4291.

\bibitem{DiMP}
Bhat G, Danelljan M, Van Gool L, et al. Learning discriminative model prediction for tracking. In: Proceedings of the IEEE/CVF International Conference on Computer Vision (ICCV); 2019 Oct 27--Nov 2; Seoul, Korea. Los Alamitos (CA): IEEE Computer Society; 2019. p. 6182--6191.

\bibitem{MixFormer}
Cui Y, Jiang C, Wang L, et al. MixFormer: end-to-end tracking with iterative mixed attention. In: Proceedings of the IEEE/CVF Conference on Computer Vision and Pattern Recognition (CVPR); 2022 Jun 19--24; New Orleans, LA. Los Alamitos (CA): IEEE Computer Society; 2022. p. 13608--13618.

\bibitem{OCL-RPF}
Ye H, Zhao J, Zhan Y, et al. Person re-identification for robot person following with online continual learning. IEEE Robot Autom Lett. 2024;9(11):9151--9158.
\end{thebibliography}

% ----- Publications page (expected to be 1 page) -----
\thesisonly{%
    \publicationspage{%
    \begin{enumerate}
       \item Your related publication 1.
       \item Your related publication 2.
    \end{enumerate}
    }
}

\end{document}


% --- Structural mapping note -------------------------------------------------
% For a single-source workflow (thesis chapters -> journal sections), prefer:
%   \kitchapter{...}, \kitsection{...}, \kitsubsection{...}
% In [ar] mode, kitinteract also enables automatic level mapping by default:
%   \chapter -> \section, \section -> \subsection, ...
% Disable with: \documentclass[ar,nomaplevels]{kitinteract}
